\section*{Ziel des Versuches}

Die Brownsche Bewegung eines Polystyrolpartikels soll unter dem Mikroskop beobachtet werden. Anhand einer Bildsequenz soll die Ortsveränderungen eines Partikels bestimmt und aus diesen die Diffusionskonstante und die Boltzmann-Konstante berechnet werden.\\
Die Brownsche Bewegung wurde 1826 von R. Brown entdeckt, konnte aber erst 1905 durch Einstein erklärt werden. Er beschrieb die Zitterbewegung der Teilchen als Resultat der Stöße der umgebenden Flüssigskeitsmoleküle.

\section*{Hinweise zur Vorbereitung}
Die Brownsche Bewegung eines Polystyrolpartikels in Wasser wird unter dem Mikroskop beobachtet.
\begin{itemize}
  \item Wie groß ist ein Wassermolekül und wie viele Größenordnungen liegen zwischen ihm und einem Polystyrolpartikel mit einem Radius von $1 \mu$m?
  \item Wie bewegen sich zwei kugelförmige Körper nach einem Stoß, wenn ihre Massen sehr unterschiedlich sind und der schwere Körper ruht?
  \item Was ist eine Normalverteilung/Gaußverteilung? Was beschreiben ihr Mittelwert und ihre Standardabweichung?
  \item Was bedeuten die Größen in der Einstein-Smoluchowski-Gleichung (\ref{eq:einstein}) und wie kann die Gleichung verstanden werden?
  \item Was bedeuten die Größen in der Gleichung für die Diffusionskonstante (\ref{eq:diff}) und wie kann die Gleichung verstanden werden?
  \item Es steht eine Simulation zur Verfügung, bei der drei Parameter verändert werden können, die Anzahl der Teilchen, die Zeitspanne und die Anzahl der Stöße pro Sekunde. Die Veränderung der Zeit und der Anzahl der Stöße ist äquivalent: Anstatt die Simulation $10$ Sekunden mit $30\times1$ Stößen laufen zu lassen, können auch $30\times10$ Stöße pro Sekunde für $1$ Sekunde durchgeführt werden. Nach einem Lauf wird ein Diagramm geplottet, indem die Schrittweite gegen eine Normalverteilung aufgetragen ist.\\
  Sollte die Simulation nicht funktionieren, kann sie auf dem Rechner am Arbeitsplatz ausprobiert werden.
  \item Was besagt das Ergodentheorem? Wie kann es mithilfe der Simulation überprüft werden?
\end{itemize}
