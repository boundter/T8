\section*{Ziel des Versuches}

Sie werden die Brownsche Bewegung eines Polystyrolpartikels in Wasser unter dem Mikroskop beobachten. Die Brownsche Bewegung beschreibt die ``Zitterbewegung'' eines Teilchens als Resultat der Stöße der umgebenden Flüssigskeitsmoleküle. Sie werden  die Ortsveränderungen eines Partikels anhand einer Bildsequenz mit einem Mikroskop bestimmen und aus diesen die Diffusionskonstante, die Boltzmann-Konstante und die Avogadro-Konstante berechnen.\\
Die Brownsche Bewegung wurde 1826 von R. Brown entdeckt, konnte aber erst 1905 durch Einstein erklärt werden.

\section*{Hinweise zur Vorbereitung}
Lesen Sie den theoretischen Hintergrund und die Literatur. Beantworten Sie die folgenden Fragen.

\begin{itemize}
  \item Wie groß ist ein Wassermolekül und wie viele Größenordnungen liegen zwischen ihm und einem Polystyrolpartikel mit einem Radius von $1 \mu$m?
  \item Wie bewegen sich zwei kugelförmige Körper nach einem Stoß, wenn ihre Massen sehr unterschiedlich sind und der schwere Körper ruht?
  \item Was ist eine Normalverteilung/Gaußverteilung? Was beschreiben ihr Mittelwert und ihre Standardabweichung?
  \item Was bedeuten die Größen in der Einstein-Smoluchowski-Gleichung (\ref{eq:einstein}) und wie kann die Gleichung verstanden werden?
  \item Was bedeuten die Größen in der Gleichung für die Diffusionskonstante (\ref{eq:diff}) und wie kann die Gleichung verstanden werden?
  \item Was besagt das Ergodentheorem? Wie kann es überprüft werden?
\end{itemize}
