\section*{Ziel des Versuches}

Die Brownsche Bewegung eines Polystyrolpartikels soll unter dem Mikroskop beobachtet. Anhand einer Bildsequenz soll die Ortsveränderungen eines Partikels bestimmt und aus diesen die Diffusionskonstante und die Boltzmannkonstante berechnet werden. Die Brownsche Bewegung wurde 1826 von R. Brown entdeckt, konnte aber erst 1905 durch Einstein erklärt werden. Er beschrieb die Zitterbwegung der Teilchen als Resultat der Stöße der umgebenden Flüssigskeitsmoleküle.

\section*{Hinweise zur Vorbereitung}
Die Brownsche Bewegung eines Polystyrolpartikels in Wasser wird unter dem Mikroskop beobachtet.
\begin{itemize}
  \item Wie groß ist ein Wassermolekül und wie viele Größenordnung liegen zwischen ihm und einem Polystyrol-Partikel mit einem Radius von $1 \mu$m?
  \item Wie bewegen sich zwei kugelförmige Körper nach einem Stoß, wenn ihre Massen sehr unterschiedlich sind?
  \item Was ist eine Normalverteilung/Gaußverteilung? Was beschreiben der Mittelwert und die Standardabweichung an ihr?
  \item Was bedeuten die Größen in der Einstein-Smoluchowski-Gleichung (\ref{eq:einstein}) und wie kann die Gleichung verstanden werden?
  \item Was bedeuten die Größen in der Gleichung für die Diffusionskonstante (\ref{eq:diff}) und wie kann die Gleichung verstanden werden?
  % TODO: Expand on simulation
  \item Was besagt das Ergodentheorem? Wie kann es mithilfe der Simulation überprüft werden?
\end{itemize}
