


\section*{Auswertung}

Zur Auswertung wird die lange Messreihe für ein Teilchen als viele kurze random walks betrachtet. Mit dem Ergodentheorem kann das Ergebnis gleichgesetzt werden mit der langen Messreihe.
Sämtliche Rechnungen sollen für $x$ und $y$ durchgeführt werden, im weiteren wird immer $x$ benutzt, aber die Auswertung für $y$ funktioniert analog.

\begin{enumerate}
\item Als erstes soll die Bewegung des Teilchens in einem $x$-$y$-Diagramm dargestellt werden. Hieran ist kurz zu erklären wie die Bewegung während der Messreihe verlief und welche Ergebnisse zu erwarten sind.

\item Sämtliche Messwerte müssen von Pixel in Meter umgerechnet werden. Dazu wird verwendet, dass zwischen zwei Skalenstrichen des Objektmikrometers $10\, \mu$m liegen.

\item Aus den Messdaten wird die Verschiebung $\Delta x$ zwischen zwei Bildern bestimmt. Für diese wird dann der Mittelwert und die Standardabweichung bestimmt. Was ist für den Mittelwert zu erwarten?

\item Die $\Delta x$ sollen in einem Histogram dargestellt und mit der Normalverteilung
\begin{equation}
  f(\Delta x, \mean{\Delta x}, \sigma) = \frac{1}{\sqrt{2 \pi \sigma^2}} e^{- \frac{(\Delta x - \mean{\Delta x})^2}{2 \sigma^2}}
\end{equation}
verglichen werden. Um die Skala anzupassen muss die Normalverteilung mit der Anzahl der Messpunkte multipliziert werden.

\item Aus $\sigma(\Delta x)$ soll die Diffusionskonstante mithilfe von Gleichung (\ref{eq:diff}) berechnet werden. Die Zeit $t$ entspricht dem Zeitintervall zwischen zwei Bildern.

\item Die berechnete Diffusionskonstante kann dann in die Gleichung (\ref{eq:einstein}) eingesetzt werden um die Boltzmannkonstante zu bestimmen. Diese ist mit dem Literaturwert zu vergleichen. Der Radius eines Polystyrol-Kügelchens beträgt $1\,\mu$m.
\end{enumerate}
