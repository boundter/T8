\section*{Auswertung}

\textbf{\underline{Hinweis:}}\textit{ Nicht alle nachfolgenden Fragen müssen während des Versuches bearbeitet werden. Konzentrieren Sie sich im Experiment auf die für die Messungen relevanten Aufgaben. \textcolor{blue}{Blau markierte Abschnitte} sind für die Auswertung relevante, weiterführende Fragen/Aufgaben und sollten spätestens im Protokoll von Ihnen diskutiert werden.}\newline

\subsection*{Teil 1 - Simulation}

\begin{enumerate}

  \item Die Ergebnisse der Simulation sollen beschrieben werden. Die Qualität der unterschiedlichen Simulationen soll eingeschätzt werden.
  \textcolor{blue}{Was ist für die Qualitäten der verschiedenen Simulationen nach dem Ergodentheorem zu erwarten?}

\end{enumerate}

\subsection*{Teil 2 - Experiment}

Sämtliche Rechnungen sollen für $x$ und $y$ durchgeführt werden, im weiteren wird immer $x$ benutzt, für $y$ funktioniert die Auswertung analog.
\textcolor{blue}{An welchen Punkt der Auswertung wird das Ergodentheorem verwendet?}

\begin{enumerate}

  \item Als erstes soll die Bewegung des Teilchens in einem $x$-$y$-Diagramm dargestellt werden. Hieran ist kurz zu erklären, wie die Bewegung während der Messreihe verlief und welche Ergebnisse zu erwarten sind.

  \item Sämtliche Messwerte müssen von Pixel in Meter umgerechnet werden. Der hierfür notwendige Massstab wird durch das Objektmikrometer ermittelt (siehe Experiment - Teil 2 Punkt 2).

  \item Aus den Messdaten wird die Verschiebung $\Delta x$ zwischen zwei Bildern bestimmt.

  \item Die $\Delta x$ sollen in einem Histogram (siehe Histogramme mit SciDAVis) dargestellt und mit der Normalverteilung (\ref{eq:gauss}) verglichen werden. Daraus ergeben sich der Mittelwert und die Standardabweichung $\sigma$. \textcolor{blue}{Was ist für den Mittelwert zu erwarten?}

  \item Aus $\sigma$ soll die Diffusionskonstante mithilfe von Gleichung (\ref{eq:diff}) berechnet werden. Die Zeit $t$ entspricht dem Zeitintervall zwischen zwei Bildern.

  \item Die berechnete Diffusionskonstante kann dann in die Gleichung (\ref{eq:einstein}) eingesetzt werden um die Boltzmann-Konstante $k_B$ zu bestimmen. Diese ist mit dem Literaturwert zu vergleichen. Aus der Boltzmann-Konstante soll die Avogadro-Konstante $N_A$ bestimmt werden. Auch diese ist mit dem Literaturwert zu vergleichen. \textcolor{blue}{Was ist die größte Fehlerquelle in dem Versuch? Ist sie systematisch oder zufällig?}\\
  Der Radius eines Polystyrol-Kügelchens beträgt $1\,\mu$m.

\end{enumerate}


\subsection*{Hinweise zur Auswertung}

\subsubsection*{Histogramme mit SciDAVis}
Eine gute Grundlage für den Umgang mit SciDAVis, ``Auswertung von Messdaten'', kann bei den Unterlagen des Grundpraktikums gefunden werden.\\
Die Daten können mit \verb|Strg + c| und \verb|Strg + v|  in die Tabellen kopiert werden. Zuerst müssen die Differenzen $\Delta x$ und $\Delta y$ bestimmt werden, dazu wird jedes Element in einer Spalte von seinem Nachfolger abgezogen. Wenn der Name der Spalte \verb|1|  ist, dann können die $\Delta$ mit der Formel \verb|col("1", i) - col("1", i - 1)| berechnet werden. Vorsicht: Der erste Messwert muss aus der Betrachtung entfernt werden, da es keinen vorherigen Wert gibt um das $\Delta$ zu bestimmen. Ausserdem müssen die Messwerte von dem Objektmikrometer entfernt werden.\\
Das Histogramm ist unter \emph{Diagramm $\rightarrow$ Statistische Diagramme $\rightarrow$ Histogramm} zu finden. SciDAVis setzt automatisch eine Intervallbreite, diese ist aber meistens nicht ideal. Um selber eine zu setzen kann, unter \emph{Format $\rightarrow$ Diagramm} das Histogramm ausgewählt werden. Unter \emph{Histogrammdaten} muss das Häkchen bei \emph{Automatische Einteilung} entfernt werden. Danach können bei \emph{Intervallbreite} eigene Werte eingesetzt werden. Um den Fit zu verbessern sollten außerdem die Werte in \emph{Anfang} und \emph{Ende} verdoppelt werden. Anschließend muss die Skalierung der Achse unter \emph{Format $\rightarrow$ Skalen} angepasst werden, damit das gesamte Histogram dargestellt wird. \\
Zuletzt fehlt noch ein Fit mit der Normalverteilung. Dieser geht sehr schnell, da SciDAVis eine eingebaute Funktion dafür hat. Diese liegt unter \emph{Analyse $\rightarrow$ Kurvenanpassungsassistent $\rightarrow$ Eingebaut $\rightarrow$ GaussAmp und Haken bei ``Mit eingebauter Funktion anpassen'' setzen $\rightarrow$ Anpassen $\rightarrow$ Anpassen}. In dem \emph{Ergebnis-Log} lassen sich alle wichtigen Größen wiederfinden. Der Mittelwert ist $xc$ (Mitte) und die Standardabweichung $\sigma$ ist $w$ (Breite). Vorsicht: Der Fit war nur erfolgreich, wenn aucht \emph{Status = success} am Ende des \emph{Ergebnis-Log} ausgegeben wird.


\section*{Literatur}

\begin{enumerate}

  \item Lüders, K. \& Oppen, G.; Band 1 Klassische Physik - Mechanik und Wärme\\ (auch Bergmann/Schaefer kompakt - Lehrbuch der Experimentalphysik Band 1) \label{it:bergmann}

  \item Meschede, D.; Gerthsen Physik

\end{enumerate}
