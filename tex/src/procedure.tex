\section*{Durchführung}

\subsection*{Teil 1 - Simulation}

Es steht eine Simulation zur Verfügung, bei der drei Parameter verändert werden können, die Anzahl der Teilchen, die Zeitspanne und die Anzahl der Stöße pro Sekunde. Für die Statistik ist die Veränderung der Zeit und der Anzahl der Stöße äquivalent: Anstatt die Simulation $10$ Sekunden mit $30\times1$ Stößen laufen zu lassen, können auch $30\times10$ Stöße pro Sekunde für $1$ Sekunde durchgeführt werden. Nach einem Lauf wird ein Diagramm dargestellt, in dem die Schrittweite gegen eine Normalverteilung aufgetragen ist.\\
\begin{enumerate}

    \item Die Unterschiede der gemessenen Verteilung und der Normalverteilung sollen für drei verschiedene Messreihen verglichen werden. Jede Simulation sollte mehrmals durchgeführt werden, da die Qualität der Ergebnisse teilweise stark schwankt.

    \begin{enumerate}

      \item Eine kurze Messreihe mit wenigen Teilchen. (z.B. $1$ Teilchen über $5$ Sekunden und $30\times10$ Stöße)

      \item Eine lange Messreihe mit wenigen Teilchen. (z.B. $1$ Teilchen über $5$ Sekunden und $30\times100$ Stöße; entspricht $1$ Teilchen über $50$ Sekunden und $30\times10$ Stöße)

      \item Eine kurze Messreihe mit vielen Teilchen. (z.B. $10$ Teilchen über $5$ Sekunden und $30\times10$ Stöße)

    \end{enumerate}

\end{enumerate}


\subsection*{Teil 2 - Experiment}

Abgesehen von der Suche nach dem Objektmikrometer sollte die ganze Zeit mit der maximalen Vergrößerung gearbeitet werden. Außerdem sollte die Lichtquelle so eingestellt sein, dass das Bild nicht überbelichtet wird. \\
Zur Beobachtung wird das Programm \emph{Grab \& Measure} verwendet. Um das Live-Video zu aktivieren, muss der Button \emph{Live} auf der rechten Schaltleiste angeklickt werden.

\begin{enumerate}

  \item Zuerst muss das Objektmikrometer fotografiert werden. Dazu muss das Objektmikrometer mit der Skala nach oben auf den Objekttisch gelegt und fokussiert werden. Anschließend wird mithilfe der Software und der Funktion \emph{Grab} ein Foto von $5$ Skalenstrichen gemacht.

  \item Um die Skalierung der Abstände in den Bilder zu bestimmen, muss das Objektmirometer vermessen werden. Dazu wird das Bild durch Doppelklick in der linken Schaltfläche ausgewählt. In dem Bild wird mit einem Linksklick ein Messpunkt am äußersten linken Strich platziert. Dieser Messpunkt kann mit einem Rechtsklick \emph{Punktkoordinaten speichern} gespeichert werden. Danach muss ein zweiter Messpunkt am äußersten rechten Strich gesetzt und gespeichert werden. Beide Punkten sollten auf einer horizontalen Linie liegen. Zur Hilfe kann das Fadenkreuz mittels des \emph{Fadenkreuz-Buttons} auf der rechten Schaltleiste eingeblendet werden. Der Abstand zwischen zwei Skalenstrichen beträgt $10 \, \mu$m.


  \item Jetzt kann das Präparat angefertigt werden. Dazu wird ein Tropfen der Polystyrol-Lösung auf einen Objektträger gegeben. Links und rechts neben den Tropfen werden Deckplättchen gelegt und auf diese beiden ein weiteres, sodass der Tropfen flach gedrückt wird. Vorsicht: Ist der Tropfen zu klein, berührt er das obere Deckplättchen nicht und es kann nicht mikroskopiert werden. Ist der Tropfen aber zu groß, berührt er die äußeren Deckplättchen und es entstehen Strömungen, die die Messergebnisse verfälschen.

  \item Nachdem das Präparat angefertigt wurde, kann mikroskopiert werden. Dazu wird der Objekttisch etwas heruntergefahren und das Präparat gegen das Objektmikrometer ausgetauscht. Danach kann der Objekttisch wieder langsam hochgefahren werden. Dabei zeigen sich verschiedene Schichten. Zuerst wird eine zerkratze Schicht sichtbar, dass ist die Oberseite des Deckplättchens, danach kommt eine zerkratzte Schicht mit ruhenden schwarzen Kügelchen, dass ist die Unterseite des Plättchens. Ab hier muss der Tisch noch ein kleines Stück nach oben gefahren werden, dann sollten die Polystyrol-Partikel als kleine, zitternde schwarze Kügelchen zu sehen sein.

  \item Jetzt kann eine Messreihe aufgenommen werden. Dazu muss ein einzelnes Polystyrol-Kügelchen gefunden und in der Mitte des Bildschirms zentriert werden. In der Nähe sollte auch kein Schmutz sein, sodass die Brownsche Bewegung ungestört beobachtet werden kann. Während der Messung darf an dem Mikroskop nicht verstellt werden und es sollte möglichst auch nicht am Tisch gewackelt werden.\\
  Nach einem Klick auf das \emph{Disketten}-Symbol in der rechten Leiste und der Einstellung des Zeitintervall  auf $2 \,$s, kann die Aufnahme durch einen Mausklick auf den \emph{Start}-Button gestartet werden.\\
  Die Messreihe besteht aus $101$ Bildern, vermutlich wird sich das Kügelchen währenddessen aus dem Bild bewegen. Sollte die Messreihe aus weniger als $60$ Bilder bestehen muss sie neu gestartet werden. Je mehr Bilder aufgenommen werden, desto genauer werden die Ergebnisse.

  \item Nach der Aufnahme der Messreihe muss noch die Temperatur im Präparat gemessen werden. Dazu wird das Präparat vom Objekttisch genommen und stattdessen wird mit dem Thermometer die Temperatur im Lichtgang gemessen. Die Messung sollte etwa $10 - 20\,$s dauern.

  \item Anschließend sollte das Objektmikrometer wieder für die nächste Gruppe scharf gestellt werden.

  \item Zum Schluss müssen die Messdaten aufgenommen und gespeichert werden.
  Dazu wird ein Messpunkt in der Mitte des Polystyrolkügelchens platziert und gespeichert. Der Vorgang ist für alle Bilder zu wiederholen.

  \item Über den Menüpunkt \emph{Messen $\rightarrow$ Messwerttabelle} können die gespeicherten Werte abgerufen und gespeichert werden.

\end{enumerate}
