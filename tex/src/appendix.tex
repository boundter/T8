\section*{Anhang}

\textbf{\underline{Hinweis:}}\textit{Der Anhang ist für die Durchführung und die Auswertung des Versuches nicht notwendig. Er gibt Ausblicke auf die Methoden der Statistischen Physik und kurze mathematische Hintergründe für einige der Annahmen und Formeln aus der restlichen Versuchsanleitung.}

\subsection*{Momente}
Als Erweiterung des Mittelwerts und der Standardabweichung können die Momente definiert werden. Das m-te Moment ist gegeben durch
\begin{equation}
  \mean{\mean{x^m}} =  \mean{\left( x - \mean{x} \right)^m}
\end{equation}
Es ist schnell zu erkennen, dass $\sigma^2$ das zweite Moment von $x$ ist, es wird auch die Varianz genannt. Typischerweise werden zur Beschreibung von Zufallsprozessen immer der Mittelwert und die Varianz angegeben. Aber auch die höheren Momente spielen eine Rolle, so ist z.B. das dritte Moment die Schiefe der Verteilung.\\
Eine besondere Wahrscheinlichkeitsverteilung ist die Normalverteilung. Hier gibt es nur ein erstes und zweites Moment, alle anderen Momente sind null. Eine Normalverteilung ist also durch die Angabe der ersten beiden Momente vollständig charakterisiert. Dieser Umstand kann benutzt werden, um zu zeigen, dass die Normalverteilung einen \emph{random walk} beschreibt:\\
$x$ ist die Entfernung von der Ursprungslage nach einer Zeiteinheit und $z$ ist die Entfernung nach $n$ Zeiteinheiten. Alle Momente steigen linear mit der Zeit, je länger der \emph{random walk} dauert, desto weiter kann er sich von seinem Ausgangspunkt entfernen.
Damit ist
\begin{equation}
  \mean{z} = n \mean{x}
\end{equation}
und für die höheren Momente
\begin{equation}
  \mean{\mean{z^m}} = n \mean{\mean{x^m}}
\end{equation}
Durch Vergleich der höheren Momente mit der Varianz zeigt sich, dass sie für lange Zeitreihen verschwinden.
\begin{align}
  \frac{\mean{\mean{z^m}}}{\mean{\mean{z^2}}^{m/2}} & = \frac{n \mean{\mean{x^m}}}{\left( n \mean{\mean{x^2}} \right)^{m/2}} \\
  & = n^{1 - m/2} \frac{\mean{\mean{x^m}}}{\mean{\mean{x^2}}^{m/2}}
  \intertext{Der rechte Bruch $\mean{\mean{x^m}}/\mean{\mean{x^2}}^{m/2}$ ist konstant für konstantes $m$. Also gilt für ein $m > 2$, dass}
  \frac{\mean{\mean{z^m}}}{\mean{\mean{z^2}}^{m/2}} & \xrightarrow{\: n \to \infty \:} 0.
\end{align}
Das bedeutet, dass alle Momente $m > 2$ für eine unendliche Beobachtungsdauer verschwinden. Die einzigen Momente die nicht zu $0$ gehen, sind der Mittelwert und die Varianz, damit muss die zugrunde liegende Verteilung eine Normalverteilung sein. Der \emph{random walk} kann also mit dem Mittelwert und der Varianz vollständig beschrieben werden.

\subsection*{Diffusionsgleichung}
Anhand des \emph{random walks} kann die Diffusionsgleichung hergeleitet werden, ohne physikalische Begriffe, wie etwa Konzentration, zu verwenden, sondern nur mithilfe der Statistik. Das deutet auf die enge Verbindung des \emph{random walks} mit der Thermodynamik/Statistischen Physik hin.\\
Zuerst wird ein \emph{random walk} mit konstanter Schrittgröße $l$ betrachtet. Die Rate einen Schritt nach links zu machen ist $\alpha$ und die Rate einen Schritt nach rechts zu machen ist $\beta$. Die Rate gibt an, wie oft ein Schritt stattfindet und kann auch als eine Form der Wahrscheinlichkeit eines Schrittes interpretiert werden. Hier wird also nicht mehr zu bestimmten Zeitpunkten eine Münze geworfen, sondern die Wahrscheinlichkeiten sind erst einmal unterschiedlich.\\ Wie verändert sich die Wahrscheinlichkeit $P$ zu einer bestimmten Zeit $t$ an einem bestimmten Punkt $x$ zu sein? Sie setzt sich zusammen aus der Wahrscheinlichkeit von rechts einen Schritt nach links zu machen und auf $x$ zu landen oder umgekehrt von rechts nach links und der Wahrscheinlichkeit von $x$ einen Schritt weg zu gehen.
\begin{equation}
  \frac{\partial P}{\partial t} (x, t) = \alpha P(x + l, t) + \beta P(x - l, t) - (\alpha + \beta) P(x, t)
\end{equation}
Um die Wahrscheinlichkeit, links oder rechts von $x$ zu sein, zu bestimmen,  kann $P$ in einer Taylorreihe entwickelt werden, wobei $l$ so klein gewählt wird, dass die Reihe nach dem dritten Glied abgebrochen werden kann.
\begin{equation}
  P(x \pm l, t) = P(x, t) \pm l \frac{\partial P}{\partial x}(x, t) + \frac{l^2}{2} \frac{\partial^2 P}{\partial x^2}(x, t) \pm \frac{l^3}{6} \frac{\partial^3 P}{\partial x^3}(x, t) + \cdots
\end{equation}
Die Taylorreihe wird in die vorherige Formel eingesetzt.
\begin{equation}
  \frac{\partial P}{\partial t} (x, t) = l (\alpha - \beta) \frac{\partial P}{\partial x}(x, t) + \frac{1}{2} l^2 (\alpha + \beta) \frac{\partial^2 P}{\partial x^2}(x, t) + \frac{1}{6} l^3 (\alpha - \beta) \frac{\partial^3 P}{\partial x^3}(x, t)
\end{equation}
Bei dem Übergang zum Kontinuum $l \to 0$ werden die folgenden Terme konstant, da sich die Raten $\alpha$ und $\beta$ wegen der kürzeren Sprungweite erhöhen.
\begin{align}
  - A & = l (\alpha - \beta) \\
  B & = l^2 (\alpha + \beta)
\end{align}
Eingesetzt in die Taylorreihendarstellung verschwinden alle Terme, die noch ein $l$ beinhalten.
\begin{equation}
  \frac{\partial P}{\partial t} (x, t) = - A \frac{\partial P}{\partial x}(x, t) + \frac{1}{2} B \frac{\partial^2 P}{\partial x^2} (x, t)
\end{equation}
Diese Gleichung erinnert stark an die Diffusionsgleichung (\ref{eq:diffusion}), wenn $P$ gegen die Konzentration $c$ ausgetauscht wird. Der $A$-Term beschreibt den Drift, also eine Bewegung des ganzen Systems und verschwindet für $\alpha = \beta$ und der $B$-Term beschreibt die Diffusion. Der Faktor $1/2$ vor dem $B$ führt auch zu der Konvention in den Gleichungen (\ref{eq:diff_random}) und (\ref{eq:diff}).
